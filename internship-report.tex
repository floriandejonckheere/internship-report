\documentclass[10pt,a4paper]{article}

\usepackage[utf8]{inputenc}
\usepackage[english]{babel}
\usepackage{fancyhdr}
\usepackage{geometry}
\usepackage{graphicx}
\usepackage{tabularx}
\usepackage{wallpaper}
\usepackage{listings}
\usepackage[toc,page]{appendix}
\usepackage{pdfpages}

\usepackage[backend=biber,style=apa]{biblatex}
\addbibresource{biblio.bib}

\usepackage{xcolor, colortbl}
\definecolor{ugentblue}{HTML}{164A7C}
\definecolor{gray}{HTML}{AAAAAA}
\definecolor{lightgray}{HTML}{FAFAFA}
\definecolor{grayborder}{HTML}{CCCCCC}
\definecolor{commentgreen}{HTML}{009900}

\definecolor{titlepagecolor}{HTML}{FFFFFF}
\definecolor{titlepagetext}{HTML}{000000}
\definecolor{titlepageline}{HTML}{207BFF}
\definecolor{titlepagesubtext}{HTML}{AAAAAA}


%%% VARIABLES %%%
\newcommand{\coursename}{Internship Applied Computer Science}
\newcommand{\courseyear}{Academy year 2017-2018}
\newcommand{\documenttitle}{Internship report}
\newcommand{\authoronename}{Jasper D'haene}
\newcommand{\authoroneemail}{jasper.dhaene@gmail.com}
\newcommand{\authortwoname}{Florian Dejonckheere}
\newcommand{\authortwoemail}{florian@dejonckhee.re}

\newcommand{\hogentlogo}{hogent-logo.png}
\newcommand{\hogentheader}{header.png}

%%% STUDENTS %%%
\usepackage{ifthen}
\newcommand{\florian}[1]{\ifthenelse{\isundefined{\isflorian}}{}{#1}}
\newcommand{\jasper}[1]{\ifthenelse{\isundefined{\isjasper}}{}{#1}}

% Change this line to print a personal report
\newcommand{\isflorian}{}

%%% TABLES %%%
\def\arraystretch{1.35}
\renewcommand{\tabularxcolumn}[1]{>{\small}m{#1}}
\newcommand{\hcell}[1]{
  \cellcolor{titlepagecolor}\color{titlepagetext}\textbf{#1}
}

\makeatletter
\renewcommand\thesubsection{\@arabic\c@section.\@arabic\c@subsection}
\makeatother{}

\usepackage[hypertexnames=false]{hyperref}
\usepackage[numbered, depth=3]{bookmark}

\renewcommand{\headrulewidth}{0pt}
\pagestyle{fancy}
\fancyhf{}

\interfootnotelinepenalty=0

%%% LISTINGS %%%
\lstset{
  backgroundcolor=\color{lightgray},
  basicstyle=\footnotesize,
  commentstyle=\color{commentgreen},
  frame=single,
  framesep=7pt,
  keywordstyle=\color{blue},
  language=Java,
  numbers=none,
  numbersep=5pt,
  numberstyle=\tiny\color{gray},
  rulecolor=\color{gray},
  stepnumber=1,
  stringstyle=\color{ugentblue},
  showspaces=false,
  showstringspaces=false
}

\begin{document}
  \includepdf{cover.pdf}

  \begin{titlepage}
    %%% FOOTER %%%
    \thispagestyle{fancy}
    \fancyhf{}

    %%% TITLE PAGE %%%
    \hfill
    \begin{minipage}[t][0.9\textheight]{0.8\textwidth}
      \pagecolor{titlepagecolor}
      \noindent
      \includegraphics[width=55px]{\hogentlogo} \\[-1em]
      \color{titlepageline}
      \makebox[0pt][l]{\rule{1.3\textwidth}{1pt}}
      \color{titlepagetext}
      \par
      \noindent
      \textbf{\textsf{\coursename}} \textcolor{titlepagesubtext}{\textsf{\courseyear}}
      \vfill
        \noindent
      {\huge \textsf{\documenttitle}}
      \vskip\baselineskip
      \noindent
      \textsf{\jasper{\textbf{\authoronename} (\href{mailto:\authoroneemail}{\authoroneemail})} \\
        \florian{\textbf{\authortwoname} (\href{mailto:\authortwoemail}{\authortwoemail})} \\
        \\
        \textbf{Coordinator}: Stefaan De Cock (\href{mailto:stefaan.decock@hogent.be}{stefaan.decock@hogent.be}) \\
        \textbf{Mentor}: Esther De Loof (\href{mailto:estherdeloof@gmail.com}{estherdeloof@gmail.com})\\
        \textbf{Location}: Open Webslides, IDLab, Ghent University
      }
    \end{minipage}
  \end{titlepage}

  %%% PAGE STYLE %%%
  \nopagecolor
  \renewcommand{\footrulewidth}{0.4pt}
  \headheight 45pt
  \ULCornerWallPaper{1}{\hogentheader}
  \fancyfoot[C]{\thepage}

  %%% TABLE OF CONTENTS %%%
  \tableofcontents

  \cleardoublepage{}

  %%% DOCUMENT %%%
  \section{Preface}
    This document describes the motivation, goals and evolution of my internship at the Open Webslides project.
    It was written in collaboration with \florian{Jasper D'haene}\jasper{Florian Dejonckheere}, another student who has done his internship at the same project.
    
    The internship was successfully completed thanks to the efforts of several people.
    First, Ruben Verborgh, the original author of the educational framework the project is set in, and who is currently teaching semantic web technologies at the University of Ghent.
    Second, the person who is responsible for all administration and communication of the project within the university, Esther De Loof.
    Finally, our internship coordinator Stefaan De Cock, who followed up during the period of the internship.
    
  \section{Project and framework}
    % Voorstelling van de stageplaats/afdeling

    The internship will be held at IDLab\footnote{\href{https://www.ugent.be/ea/idlab/en}{https://www.ugent.be/ea/idlab/en}} (Internet Data Lab), a core research group of imec and Ghent University.
    Ruben Verborgh, professor Semantic Web technology at Ghent University and researcher within IDLab, is the original initiator of the project where the internship will take place.
    He proposed the idea for \textit{open webslides}, using modern webtechnologies to create course material and enable students to interact more with the teacher and with each other.
  In 2016 the Open Webslides\footnote{\href{https://openwebslides.github.io}{https://openwebslides.github.io}} project won the UGent Innoversity Challenge by convincing the university of the project's vision on the future of education: open and web based.

    Two years later and one year after the kickoff of the development on a co-creation platform for webslides, the project now counts people from many faculties within the UGent, multiple universities and even a wider international reach using an Erasmus+ project funded by the European Union.

    \florian{
      I have chosen this project because of several reasons.
      First, I have already been working on the project as an active developer since Feburary 2017.
      This allowed me to jump right into the project without any kind of learning curve or impediment.
      Second, since I am the main backend developer and have designed and implemented a large part of the backing logic, it is beneficial for the project to allow me to continue developing the platform.
      Finally, the project stays within the domain of interest I have set out for myself -- Ruby on Rails and dynamic web applications.

      My role in the project concerns more than just developing the backend software.
      It also reaches into the frontend stack, effectively making me a full-stack developer.
      In February 2018 the development team decided to reimplement a large part of the frontend stack, causing an imbalance between frontend and backend.
    }
  
    \jasper{
      % TODO
    }

  \section{Responsibilities and assignments}
    % Beschrijving van de opdracht, inclusief (business) doelstellingen duidelijk formuleren

    The current roadmap and milestones for the Open Webslides project are tailored to the two internship students working on the development.
    By the end of their internship (end of May 2018), the frontend part of the platform will have been reimplemented, providing a basic yet adequately functional interface for the Open Webslides platform.
    One of the important aspects of this reimplementation is the presence of a strong developer mentality and code foundation, based on adequate testing coverage, integration tests, pull request reviews and internal conventions.
    The full description of the principles and conventions is written down in the project charter\footnote{\href{https://docs.google.com/document/d/1k5-6LX3VHIK5Jqs8hX9gVvC9cr81brG-fZm1qaF290w}{https://docs.google.com/document/d/1k5-6LX3VHIK5Jqs8hX9gVvC9cr81brG-fZm1qaF290w}}.
    The main focus of this development phase is on the slides editor: a user should be able to create and modify topics, making use of all the interactive and semantic utilities the web has to offer.\\

    \florian{
      Since I have been involved in the project since the beginning of the platform development, I am the main responsible person for defining the milestones, determining the product backlog and leading the development team on a day-to-day basis.

      However, my primary role in the development team is backend developer.
      I am responsible for developing the backend server, implemented as a REST API server.
      Designing and maintaining the supporting server infrastructure is also one of my responsibilities.
      I also provide supporting frontend development, since the main workload of this development phase lies in the frontend stack.

      Finally, I am also employed as technical administrator for the CoCOS Erasmus+ project.
      This project researches co-creation using open source tools, and Open Webslides is one of the tools it will use in the future.
      Supporting the project website and the interactive learning environment (Moodle) are examples of responsibilities concerning the CoCOS project.
    Keeping the Open Webslides project roadmap, milestones and expectations synced up with the intellectual outputs of the Erasmus+ project are important tasks as well.
    }
  
    \jasper{
      % TODO 
    }


  \section{Completed assignments}
    % Uitgewerkte opdracht(en) (dit bevat indien van toepassing analyse, ontwerp en implementatie)

    \florian{
      \begin{itemize}
        \item \textbf{Deployment}: implementation of the web application's runtime environment. 
              The deployment procedure uses Docker\footnote{\href{https://www.docker.com/}{https://www.docker.com/}} and Docker Compose\footnote{\href{https://docs.docker.com/compose/}{https://docs.docker.com/compose/}} to create a reproducible and stable build of the product.
        \item \textbf{REST API}: the backend server was mainly written during the previous years, however due to conceptual changes some parts had to be rewritten.
              Examples of these changes include naming changes, the move from storing plain HTML as course content in the backend to a controlled JSON or YAML format, integrating UGent CAS into the system.
        \item \textbf{Authentication}: since the authentication module in the backend was already available, frontend integration went smoothly.
              The platform uses a mechanism of JSON Web Tokens, authenticating every request with the API server while storing the renewed token.
              The authentication module in the frontend includes user signup, sign in and reset password functionality.
        \item \textbf{Slide rendering}: course content is structured in a hierarchy.
              From this hierarchy slides and full-text pages are then derived.
              Together with another developer on the project, I wrote the slide rendering logic.
        \item \textbf{CoCOS}: a major assignment was the setup and maintenance of the CoCOS project site\footnote{\href{https://www.cocos.education/}{https://www.cocos.education/}} and digital learning environment\footnote{\href{https://my.cocos.education/}{https://my.cocos.education/}} in collaboration with international partners.
              The project website comes in the form of a self-hosted WordPress application, while the digital learning environment is a self-hosted software package called Moodle\footnote{\href{https://moodle.org/}{https://moodle.org/}}.
      \end{itemize}
    }
  
    \jasper{
      % TODO 
    }

  \section{Personal reflection}
    % Eindreflectie
    
    \florian{
      I have learned a lot of skills during this internship, both soft and hard skills.
      As one of the main programmers and the developer longest working on the project already, some decisions and issues end up in my workload.
      I have learned to manage people better, give better presentations and product demos.
      
      On the technology side, I have learned a lot about the React framework and the surrounding ecosystem, learning to program using a completely different paradigm than what I'm used to.
      This internship has also given me more perspective about my future and how I want to spend the next few years.
    }
  
    \jasper{
      % TODO
    }

  \section{Conclusion}
    % Algemeen besluit:
    %   Business doelstellingen (al dan niet gehaald + waarom)
    %   Persoonlijke doelstellingen (al dan niet gehaald + waarom)

    The main business goal of the internship period was reimplementing part of the existing platform, using proper code standards and community best practices.
    The deliverable to be finished in June is a basic platform, which includes the following features:
    
    \begin{itemize}
      \item \textbf{Platform functionality}: users can sign up, sign in and edit their profiles on the web application
      \item \textbf{Topic management}: users can create, edit and delete new topics
      \item \textbf{Basic content editor}: users can add, edit and remove new content items -- headings, paragraphs, lists, images, ...
    \end{itemize}
  
    All of these features were finished by the end of the internship.
    
  \section*{Glossary}

  \printbibliography[heading=bibintoc]

  \begin{appendices}
    \section{Internship journal}
    \florian{
      The internship journal was written after the end of every sprint, which usually consists of two weeks.
      The entries were submitted using the internship tool (\href{http://tinfbo1.hogent.be/}{http://tinfbo1.hogent.be/}).

      \subsection*{Sprint 1}
        End of the first sprint. All backlog tasks included in this sprint were successfully completed.

        \begin{itemize}
          \item Setup of the frontend repository
          \item Mockups and wireframes of the application framework and editor
          \item Cleanup of old backend code
          \item Infrastructural support of the CoCOS-project
        \end{itemize}

      \subsection*{Sprint 2}
        End of second sprint. Finished backlog items:

        \begin{itemize}
          \item Refactoring of Decks (Topics) API
          \item Initial layout homepage
          \item Initial library layout
          \item Data model state validation
        \end{itemize}

        Ongoing tasks:

        \begin{itemize}
          \item Redesign of the CoCOS project site
        \end{itemize}

        Next sprint goal: Finishing a demo-able product for the CoCOS conference in Madrid. This includes:

        \begin{itemize}
          \item Rudimentary rendering of topic content as slides
          \item Editable content hierarchy
          \item Expand library pages
          \item Design Open Webslides branding
        \end{itemize}

        Planning for next week:

        \begin{itemize}
          \item Start coupling frontend and backend: data has to retrieved from and be stored in the backend instead of local state
          \item Integrate authentication (signin/signup pages, REST API authentication and authorization)
          \item Start designing platform branding: the platform needs its own house style, colours, logo
        \end{itemize}

        Final sprint demo and meeting with the stakeholders and product owner was very productive.
        Feedback on mockups and the current application was provided and will be integrated in the next sprint (3 weeks instead of 2).

      \subsection*{Sprint 3}
        End of third sprint (19/03 - 05/04).
        Finished backlog items:

        \begin{itemize}
          \item Redesign CoCOS project site
          \item Integrate Semantic UI LESS version
        \end{itemize}

        Ongoing tasks:

        \begin{itemize}
          \item Authentication pages in frontend
          \item API coupling frontend-backend
        \end{itemize}

        Next sprint goal:

        \begin{itemize}
          \item Editor interface (not necessarily coupled to backend)
          \item Platform branding, styling
          \item Working online demo
          \item Authentication (sign in, sign up) should also be working
          \item Slide rendering: very important as Anneleen is depending on the development of this feature to continue her master's dissertation
        \end{itemize}

        Final sprint demo and retrospective taught us that development is not going as fast as expected, however we should still be able to meet the milestone at the end of the internship: having a working content editor.

        The past week I went to Madrid to meet up with the consortium of the CoCOS Erasmus+ project to discuss the further development and milestones up until February 2019.
        From now until summer the project is on track, but development efforts will have to continue in order to meet the deadlines in August and February.

      \subsection*{Sprint 4}
        End of fourth sprint. Finished items:

        \begin{itemize}
          \item API coupling
          \item Demo instance
          \item Refactoring
        \end{itemize}

        Planning for next sprint (very short sprint due to holidays):

        \begin{itemize}
          \item Persistence of topics in backend
          \item Ongoing development of the editor
        \end{itemize}

      \subsection*{Sprint 5}
        End of fifth sprint. Finished items:

        \begin{itemize}
          \item Topics/ContentItems API coupling
          \item Refactoring, solving issues
        \end{itemize}

        Backlog for next sprint:

        \begin{itemize}
          \item Prepare for final sprint demo
          \item Ongoing development of the editor
          \item Fixing various issues
        \end{itemize}

      \subsection*{Sprint 6}
        End of sixth sprint. Finished items:
        
        \begin{itemize}
          \item Topics/ContentItems testing
          \item Deployment framework
          \item Refactoring, issues
        \end{itemize}
    }

    \section{Reflection reports}
    
    \includepdf[pages=-]{reflectieformulier-1.pdf}
    \includepdf[pages=-]{reflectieformulier-2.pdf}
    \includepdf[pages=-]{reflectieformulier-3.pdf}
  \end{appendices}

\end{document}
